\documentclass[10pt,final,journal]{IEEEtran}
\usepackage{xcolor} % Allows to make use and define many colors.
\usepackage[numbers]{natbib}
\usepackage{graphicx}
\usepackage{hyperref} 
\usepackage{caption}


\newcommand{\todo}[1]{\color{red}\_ToDo: #1 \color{black}}
\graphicspath{{./img/}}
\title{Continuous room localization using painting detection}
\author{Bert De Saffel, Timothy Thiecke}
\begin{document}
	\maketitle
	\begin{abstract}
		This paper describes a localization method based on paintings from a museum, in this particular case The Museum of Fine Arts, Ghent. The method consists of three parts. The first part is painting segmentation which attempts to detect a painting from an arbitrary video frame using painting contours and a bounding box and transforms this painting into a standard format such that it can be used for analysis. The second part considers the transformed image and uses a brute-force ORB  matcher to detect key features and descriptors. These features are matched against a database using a linear lookup method. The third step uses the information of the painting, which contains the room it is located in, to mark it on a ground plan of the museum.	The painting segmentation achieves an accuracy of 47.1\% while the matcher achieves 46.67\%.
	\end{abstract}

	\input{./tex/Introduction.tex}
	\section{Painting Detection}
	\todo{ook dingen uitleggen die niet werkte}
	\begin{itemize}
		\item \todo{vanishing points}
		\item \todo{hough transformatie}
		\item \todo{lijn intersectie}
		\item \todo{gabor filter}
		\item \todo{local binary patterns}
	\end{itemize}

	\todo{gebruik ook afbeeldingen}

	\subsection{Painting Segmentation}
	The first step of the algorithm is the segmentation of an arbitrary video frame to detect a painting. A typical painting contains the art on its own enclosed by a painting frame. This painting frame causes a strong change in environment, increasing the effectiveness of an edge detector. Extracting the edges with the Canny edge detector yields a first indication of where a painting might be. If the full painting frame is visible on the video frame, its contour can be calculated using \cite{SUZUKI198532} which returns a vector of points for each contour. We consider only contours which have four points.

	It is possible that multiple paintings exist on a single frame. However, the algorithm's goal is to detect in which room the user is located. Multiple paintings on the same wall belong to the same room. Hence, the algorithm will try the matching procedure in a later phase on one detected painting with the largest surface area. The remaining paintings are ignored but may be the result of the segmentation step in any of the following video frames.

	The detected painting is then transformed through a homography to a rectified version which serves as the input of the following stage.

	\subsection{Feature Detection and Matching}
	Feature detection and extraction is applied to the extracted painting from the segmentation phase and will be matched with an image from the database. Feature extraction is done with ORB.

	Matching is done by invoking a matching procedure between the extracted keypoints and the keypoints of the database images. A match between descriptors is defined by its distance metric. The lower this number, the more likely that the match is valid. We calculate the sum of all matches and sort the matches between the source and database images by this sum. The first entry in this collection of matches is the image that is estimated to be a match for what is currently seen on-screen. 

	Additional meta-data is associated with the matched image and is used for the next phase.

	\subsection{Path Tracking}
	Once a painting is identified and matched, it can be localized on the ground plan. To achieve this, the ground plan is converted into a directed graph. The nodes of this graph are the rooms of the museum and the edges define the connections between rooms. When a user starts recording paintings, the matching algorithm will be performed on each frame and a location will be found. The graph is able to mark nodes in three distinct ways. A green node is the start of the path, an orange node is an intermediate path and the blue node is the end of the path. The path ends when the user stops recording. The path direction is also visualized by coloring the corresponding edges green. Note that when a cyclic path occurs which was walked in both directions, information of order is lost.


	To illustrate the path tracking algorithm,  a small segment consisting of rooms 1, 2, 3, 4, 5, 6, 7 and 8 are converted into such a graph and is show on figure \ref{fig:groundplan_msk_simple_graph}.


	%digraph G {
	%	2[fontcolor=white, fillcolor=green, style=filled]	
	%	4[fillcolor=orange, style=filled]
	%	5[fillcolor=orange, style=filled]
	%	7[fillcolor=orange, style=filled]
	%	6[fontcolor=white,fillcolor=blue, style=filled]
	%	1 -> 2
	%	2 -> 1
	%	2 -> 3
	%	2 -> 4[color=green]
	%	2 -> 5
	%	3 -> 2
	%	4 -> 2
	%	4 -> 5[color=green]
	%	4 -> 7
	%	4 -> 8
	%	5 -> 7[color=green]
	%	7 -> 6[color=green]
	%	6 -> 7
	%	7 -> 8
	%	7 -> 5
	%	7 -> 4
	%}

	\begin{figure}
		\includegraphics[width=\linewidth]{groundplan_msk_simple_graph}
		\caption{Path tracking using a graph. The green marks the path's start, orange nodes the visited rooms, and blue the last one visited. Edges in green denote help visualize the path.}
		\label{fig:groundplan_msk_simple_graph}
	\end{figure}

	\subsection{Database}
	The database consists of 688 images of various paintings and sculptures in the museum. In this work we only focus on the paintings of this dataset. The paintings were extracted from two different camera's: a Nokia 7 plus and a Samsung A3. Each image also contains the room in which it resides as meta-data.  

	To reduce the load time of this database, a prebuilding stage was implemented. This stage reduces each image to a collection of interest points and corresponding descriptors for these interest points as generated by the ORB \cite{Rublee2011} algorithm. 
	
	\section{Results}
\label{sec:results}
This section describes the experimental results of the final recognition system. It includes the performance of the individual steps: painting segmentation, the matching algorithm and room localization. 


\subsection{Dataset}
The main dataset consists of 688 pictures of all art items in the museum which functions as a database. These pictures are taken at eye-level height and each picture contains one or multiple art items. One part of the dataset is taken with a Nokia 7 Plus camera, which offers a base resolution of 3024 by 4032 pixels and the other part is taken with a Samsung A3 2016, which offers a base resolution of 3096 by 4128 pixels. All pictures in the dataset are compressed to 1000 x 1000 pixels. To reduce the load time of this dataset, a prebuilding stage was implemented. This stage reduces each image to a collection of interest points and corresponding descriptors for these interest points as generated by the ORB \cite{Rublee2011} algorithm. 

\subsection{Testset}
Apart from the dataset, two different testing sets exist to evaluate the algorithm. The first testing contains still images of various shots at the museum. This dataset contains more difficult examples such as steep angles or hard to detect paintings. A random sample ($n = 30$) was selected and labeled manually. This first testset is used to evaluate the segmentation and matching part. A second testing set are various videos which emulates a person recording paintings while moving trough the museum. Similar to the first testset, a small segment (1 minute) of the video was taken and labeled manually. This last testset is used to evaluate path tracking.

\subsection{Segmentation Accuracy}
The random sample from the testset was first manually segmented to indicate a perfect segmentation. This results in 4 coordinates of a polygon which represent the ideal polygon. Afterwards, the segmentation is done automatically by the algorithm, which also gives four coordinates of a polygon. To illustrate, both polygons are shown on figure \ref{fig:painting_segmentation_validation_1}.
To measure the similarity of these two polygons, and thus the correctness of the segmentation, we first calculate the intersection area $A_i$ and the area of the ideal polygon $A_{pi}$. The ratio of $A_i$ to $A_{pi}$ describes the closeness of two polygons with 100\% being a perfect match and 0\% meaning there is no intersection at all between the two polygons. There is one case where this statistic does not work. When the ideal polygon is fully enclosed by the predicted polygon, $A_i$ will be equal to $A_{pi}$, resulting in a fake perfect match. To prevent this, the roles of the green and red polygon are switched such that we now consider the area of the predicted polygon, $A_{pp}$, instead of $A_{pi}$.

Using this metric, the segmentation method achieves 88.57\% correctness score. Due to the use of the Canny edge detector, the segmentation works fairly well when the painting frame and the background wall differ greatly in color intensity. Problems start to arise when shadows are visible (figure \ref{fig:negative_case_shadow}) or when the painting frame is not fully visible. In the former case, the shadow and the background wall also differ in color intensity, resulting in an edge. These edges usually extend the painting frames and because the contour with the largest area is sought after, the segmentation step will include the shadow as part of the painting.


\begin{figure}
	\centering
	\includegraphics[width=\linewidth]{painting_segmentation_validation_1}
	\caption{A comparison of a manually selected polygon (red) and the polygon found by the segmentation algorithm (green).}
	\label{fig:painting_segmentation_validation_1}
\end{figure}
	\begin{figure}
	\includegraphics[width=0.5\linewidth]{negative_case_shadow}
	\caption{An example of a shadow underneath the painting. This usually results in the segmentation algortihm to include this shadow as part of the painting because of the strong edge.}
	\label{fig:negative_case_shadow}
\end{figure}


\subsection{Matching Accuracy}
The matching algorithm has to be evaluated manually by comparing the matcher's result. The correctness of the matching algorithm is simply the ratio of the correct matches against the false matches.

To evaluate the room localization, a sample of the video dataset was taken. The generated path is compared against the actual path.


	\todo{qualitative as well as quantitative}
	
	\todo{quantitative: graphs, tables, roc-curves, f1-scores, ...}
	
	\todo{qualititative: technisch, show where and why the method succeeds or fails, pictures of easy and difficulty cases}
	
	Because our method relies heavily on edge detection, there are cases where this could have a negative impact. In many cases, there is a shadow underneath the painting, as shown on figure \ref{fig:negative_case_shadow}.
	


	\todo{stukje hierboven ook kwalitatief?}
This subsection, we will present a qualitative analysis of our algorithm. We will discuss its strengths, flaws and, with each point, present an example case to help as a visual aid. The flaws in particular help paint a picture of what can be done better in a future iteration of the algorithm. 
e do not deny its usefulness and believe that it may be beneficial to implement it in a future version of the algorithm \todo{waar nodig?, kan zijn dat ML gebruikt wordt bij matching}. Even though the algorithm's simplicity is presented as one of its strengths, it can also be considered as one of its weaknesses, much like a double edged sword.
match and detected keypoints between two runs. However, due to not detecting different potential candidate matches between runs results in the algorithm never being able to present a different solution for the given image.
 the matching function, but still manages to find a correct match.
lied to the building stage of the database.
ion phase detects such a painting and supplies it to the matching phase. A cascade of erroneous matches and room localizations may occur. The inverse is also true, as having an entry in the data set with few keypoints can result in a painting with many keypoints being wrongfully matched with a 'flat' one.
e descriptors of the segmented painting results in a total.
ion to this problem may lie in other journals \todo{ref naar boek/conference over bag of words/large databases}. \todo{laatste paragraaf herschrijven}
	\section{Conclusion}
\label{sec:conclusion}
	\todo{overview of the most important contributions and the results, without introducing anything new}
	
	\todo{after the reader has read the paper, the reader can look at the contributions and results from a different viewpoint}
	
	\todo{statements can be made more explicit}
	
	\todo{eventueel future work}

	\bibliographystyle{IEEEtran}
	\bibliography{IEEEabrv,library}
\end{document}